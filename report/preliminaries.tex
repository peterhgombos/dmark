\section{Preliminaries}


The original Delta Correlating Prediction Tables algorithm can be seen in
algorithm \ref{alg:dcpt_orig}, as described in Grannæs et.
al.\cite{Grannaes481837}. Initially, the PC is looked up in a table of entries,
and if it is not found, a replacement entry is created. If it was in the table,
the new delta from the last address to the current is added to the list of
deltas. 


The tiered version of the DCPT exchanges the table of entries with different
tier tables, which are searched through sequentially. The TDCPT did not yield
any significant speedup, but did lower the memory footprint.


\begin{algorithm}
\footnotesize
\caption{Original DCPT \cite{Grannaes481837}}
  \begin{algorithmic}[1]
    \Procedure{DCPT}{Load address $PC$, Miss address $addr$}
      \State $r \gets \Call{TableLookUp}{PC}$ \\

      \If {$entry.pc \neq PC$}
        \State $entry.pc \gets PC$
        \State $entry.lastAddress \gets addr$
        \State $entry.deltas[\;] \gets \emptyset$
        \State $entry.lastPrefetch \gets 0$
      \ElsIf {$addr - entry.lastAddress \neq 0$}
        \State $entry.deltas[\;] \gets addr - entry.lastAddress$
        \State $entry.lastAddress \gets addr$ \\

        \State $candidates[\;] \gets \Call{DeltaCorrelation}{entry}$
        \State $prefetches[\;] \gets \Call{PrefetchFilter}{entry, candidates[\:]}$
        \State $\Call{IssuePrefetches}{prefetches[\:]}$
      \EndIf
    \EndProcedure
  \end{algorithmic}
  \label{alg:dcpt_orig}
\end{algorithm}


\subsection{Prefetcher Description}

Our prefetcher, Dynamically Tiered Delta Correlating Prediction
Tables aims to have dynamically sized tiers to improve upon the potential
pitfalls that the original TDCPT-prefetcher fell into. Additionally it will
gather some statistics about the hit-rates for the various tiers to attempt to
decide better what size the various tiers should be, and whether or not to tier
at all.

% TODO: Fill in more when the architecture becomes a bit clearer.

\subsection{Methodology}

We will be investigating the value of dynamically tiering the DCPT by
implementing the algorithm in software and testing it with a subset of the SPEC
CPU2000 benchmark.  The algorithm will be implemented in C and run on the M5
simulator, running on the Kongull cluster at NTNU.
