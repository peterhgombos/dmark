\section{Introduction}

\IEEEPARstart{T}{he} 2013 paper by Sømåen et. al~\cite{Somaen} mentions several
interesting avenues for further research and improvement in
DCPT-based~\footnotemark[1] prefetching. Among these it mentions the
possibility that "a large full size buffer would be more useful for the parts
of the program that generate proper deltas". As a follow-up, this paper
investigates the potential benefits of a hybrid scheme that switches between a
two-tiered TDCPT scheme and ordinary DCPT on demand. Depending on the degree of
reuse in it's tier 3, it is possible that some of the benefits found in pure
DCPT might be retained. Potentially, this would also have the benefit of
enhancing performance in sections that would not benefit from such a large
buffer, but where longer pattern lifetime is of greater benefit.

\footnotetext[1]{Delta-Correlating Prediction Table}

An important detail to the hybrid scheme discussed in this paper is at what
point the prefetcher should interchange between the two tiers. In the
implementation it's easy to change the tolerance of the buffer, and only when
the ratio of misses is high enough, the change happens. At exactly what point
this limit should be is examined in this paper.

