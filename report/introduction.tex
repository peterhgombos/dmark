\section{Introduction}

% Usikker på hvor denne passer inn, men tenkte å drodle ideer for å komme
% igang:
\IEEEPARstart{T}{he} 2013 paper by Sømåen et. al~\cite{Somaen} mentions several
interesting avenues for further research and improvement in
DCPT-based~\footnotemark[1] prefetching. Among these it mentions the
possibility that "a large full size buffer would be more useful for the parts
of the program that generate proper deltas". As a follow-up, this paper
investigates the potential benefits of a hybrid scheme that switches between a
two-tiered TDCPT scheme and ordinary DCPT on demand. Depending on the degree of
reuse in it's tier 3, it is possible that some of the benefits found in pure
DCPT might be retained. Potentially, this would also have the benefit of
enhancing performance in sections that would not benefit from such a large
buffer, but where longer pattern lifetime is of greater benefit.

\footnotetext[1]{Delta-Correlating Prediction Table}

%(Alternatively, we could get stale entries out earlier by reducing the size of Tier 3). 
The paper describes that "an unweighted circular buffer could lead to loss of important
delta patterns", and then goes on to consider adding further
weighting information to better decide when and what to replace. A potential source
of further information, is to collect the hit-rate for tiers 1-3, and decide based 
those what mode to operate in (i.e. what size the buffer should be, should it fall 
back to a proper, or 2-tiered approach?). For instance, if we have a very low
sustained hit-rate in tier 3, then perhaps all of tier 3 should be purged, to avoid
stale patterns in a more explicit form.

Additionally, it might be worth looking into where the "bursts of instructions that
only happen once" (Sømåen et al) comes from. Perhaps we can either create some kind
of protection against them flushing the buffer, or if indeed they correlate better
with a situation where the program is changing it's operation, and thus would be
a decent way to detect that the buffer should be flushed, then that might be a
possible improvement. A possible reasoning behind this, is that the TDCPT-scheme as
described in Sømåen et al, goes out of it's way to keep deltas for as long a period
as possible, while there might exist periods of time where you either want to
ignore new patterns (sudden bursts of non-related work, that then go back to a
previous pattern), it also lets the patterns get a bit more stale, thus letting the
risk of too aggressive prefetching increase.

Our main approach is thus looking into whether the TDCPT-scheme can indeed be
improved upon in a fashion so that it is better (atleast in some regard) than the
DCPT-scheme it set out to improve on. Our two approaches will be to look into a
dynamic partitioning of the tiers, and trying to harvest some usage statistics for
trying to change the mode of operation. The reasoning behind trying to give TDCPT
a further look into, is that it DID show atleast some improvement upon DCPT, even
though it overall fared worse than the original DCPT-scheme.
